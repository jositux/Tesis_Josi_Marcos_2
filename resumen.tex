% ADDITIONAL DECLARATIONS HERE (IF ANY)

%\vskip5mm
%Signature:
%\vskip20mm
%AUTOR
%José María Guaimas



%-----------------------------------------------------
% Resúmen 2
%-----------------------------------------------------
\addcontentsline{toc}{section}{Resúmen}
\section*{Resúmen}
En el pasado el proceso de fabricar un producto consistía en delegar el diseño o enviar planos a un especialista para que realice la manufactura, los contactos eran presenciales y mantener el proyecto bajo control obligaba a realizar viajes frecuentes. 

En la actualidad el contexto es muy diferente: El Desarrollo Colaborativo de Productos y el Co-Diseño son conceptos muy utilizados en las organizaciones, en especial, en las áreas que involucran el diseño y fabricación asistido por computadora usando técnicas CAD/CAM. Gracias a la evolución de las tecnologías web se asume la colaboración entre personas dispersas geográficamente, de diferentes campos de especialización e incluso sin formación en diseño. \vskip
Al mismo tiempo, la diversidad de conocimientos trae como consecuencia algunos problemas en la gestión de los proyectos, entre ellos: la comunicación imprecisa entre los participantes, la múltiple interpretación de ideas y la complejidad en el registro de cambios en los diseños. 

El presente trabajo propone el diseño e implementación de un prototipo de aplicación web que permite la colaboración multidisciplinaria mediante revisiones de modelos 3D, estableciendo una comunicación fluída entre los participantes que vaya más allá de la geometría. 

El prototipo llamado \textit{Colaborative CAD Application} (COCADA) utiliza el enfoque LEAN UX combinado con metodologías ágiles de desarrollo de software y el uso de tecnologías web \textit{Free Libre Open Source Software} (FLOSS)\footnote{\url{https://es.wikipedia.org/wiki/Software\_libre\_y\_de\_c\%C3\%B3digo_abierto}}. \vskip
 \vskip
El software COCADA se encuentra disponible para la descarga en \url{https://github.com/jositux/Tesis_Josi_Marcos}  bajo licencia GNU General Public License (GPL)\footnote{\url{https://www.gnu.org/licenses/gpl-3.0.en.html}}. 

\vskip5mm
Palabras Claves: Diseño Colaborativo, CPD, OpenJSCAD, Javascript, Vue.js, Node.js, LEAN UX, Agile




	
% ADDITIONAL DECLARATIONS HERE (IF ANY)

%\vskip5mm
%Signature:
%\vskip20mm
%AUTOR
%José María Guaimas




\textbf{Conceptos de Loopback}
https://yo.toledano.org/desarrollo/loopback-conceptos-basicos/