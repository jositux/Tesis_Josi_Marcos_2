IMPORTANTE PARA JUSTIFICAR

https://www.uxpin.com/studio/blog/forget-tedious-documentation-prototype-requirements-instead/
http://www.agilemodeling.com/essays/agileDocumentationBestPractices.htm

http://jmbeas.es/guias/manual-del-buen-dueno-de-producto-product-owner/

Chaos report
http://www.laboratorioti.com/2016/05/16/informe-del-caos-2015-chaos-report-2015-bien-mal-fueron-los-proyectos-ano-2015/

http://repository.unimilitar.edu.co/bitstream/10654/13523/1/Articulo_final_NELSON_RIVERA.pdf

https://www.forbes.com/sites/bernardmarr/2016/09/13/are-these-the-real-reasons-why-tech-projects-fail/#6646af027320

https://www.youtube.com/watch?v=xh7P1WNV0ME

PAPER colaborative
http://www.inf.ufrgs.br/~oliveira/pubs_files/Fan_Oliveira_M_K_sketch.pdf






https://en.wikipedia.org/wiki/Constructive_solid_geometry

http://lsi.ugr.es/~cad/teoria/Tema5/RESUMENTEMA5.PDF

https://www.solidworks.es/sw/products/3d-cad/3d-solid-modeling.htm

http://cad3dconsolidworks.uji.es/CAD3DSW1_T1_Modelado_Cap01.pdf
 


General
Referencias
http://www.paramach.com/features/cadcam-on-demand/

Patente impresora 3D
https://www.google.com/patents/US5121329

https://es.wikipedia.org/wiki/STL

https://es.wikipedia.org/wiki/OpenSCAD

https://es.wikipedia.org/wiki/Interacci%C3%B3n_persona-computadora

http://joostn.github.io/OpenJsCad/
https://www.npmjs.com/package/openscad-openjscad-translator



Interface
https://medium.com/interactive-mind/the-only-ux-reading-list-ever-d420edb3f4ff#.d1vzepi7t

ttp://learning.media.mit.edu/content/publications/EA.Piaget%20_%20Papert.pdf

http://neoparaiso.com/logo/seymour-papert.html

http://wiki.laptop.org/go/Seymour_Papert

https://www.uam.es/personal_pdi/stmaria/jmurillo/InvestigacionEE/Presentaciones/Curso_10/Inv_accion_trabajo.pdf

http://www.tinia.org/2012/10/master-thesis-proposal-adapting.html

https://books.google.com.ar/books?id=Bk5Uv0Aiis0C

/* WebGL */
http://www.tinia.org/2012/10/master-thesis-proposal-adapting.html
http://evanw.github.io/csg.js/

https://sketchfab.com/
http://chimera.labs.oreilly.com/books/1234000000802/ch01.html#the_browser_as_platform

Conceptos Interface HM
https://medium.com
https://uxplanet.org
http://aipo.es/

https://www.copadata.com/es-es/soluciones-hmi-scada/interfaz-hombre-maquina-hmi/
http://www.grihotools.udl.cat/mpiua/user-experience/

http://www.ifml.org/

https://en.wikipedia.org/wiki/Douglas_Engelbart

https://scotch.io/tutorials/creating-a-single-page-todo-app-with-node-and-angular

Diseño de interacción
http://www.ixda.com.ar/que-es-diseno-interaccion/

DISEÑO ITERATIVO
https://www.nngroup.com/articles/iterative-design/

http://www.ijdesign.org/ojs/index.php/IJDesign/article/view/387/240

UX
https://medium.com/espanol/los-8-mejores-libros-para-aprender-ux-en-2016-e1f04303be31#.1ijmw558w

http://madeinmeshoes.com/en/shoe-creator-for-women
https://www.blend4web.com/en/demo/


ESTANDARES
http://albertolacalle.com/hci_estandares.htm

SOFTWARE INSPIRED
http://beta.speckle.xyz/
https://github.com/didimitrie/future.speckle
https://chrome.google.com/webstore/detail/3dview/hhngciknjebkeffhafnaodkfidcdlcao
http://www.modelo.io/


Testeo de usabilidad
https://medium.com/bridge-collection/a-guerilla-usability-test-on-dropbox-photos-e6a1e37028b4#.gnibpve7m

https://medium.com/chat-bots/usability-heuristics-for-bots-7075132d2c92#.j4qccv1xg

https://www.nngroup.com/articles/f-shaped-pattern-reading-web-content/

https://www.nngroup.com/articles/ten-usability-heuristics/





CPD
LAS NUEVAS HERRAMIENTAS PARA EL DESARROLLO DE PROYECTOS INTENSIVOS EN INGENIERÍA CON LA COLABORACIÓN DE VARIAS EMPRESAS
http://www.aeipro.com/files/congresos/2004bilbao/ciip04_0656_0666.1253.pdf

Emergence in collaborative computer-aided design
https://www.aaai.org/Papers/Workshops/1993/WS-93-07/WS93-07-020.pdf

Collaborative product developmet. A collaborative desicion making-approach
http://www.diva-portal.org/smash/get/diva2:208401/FULLTEXT03.pdf

Managing collaborative product developmet
https://www.diva-portal.org/smash/get/diva2:120694/FULLTEXT01.pdf

CONCEPTOS CAD COLABORATIVO
https://en.wikipedia.org/wiki/Collaborative_product_development
https://en.wikipedia.org/wiki/Visualization_(computer_graphics)#Product_visualization
https://en.wikipedia.org/wiki/CAD_data_exchange

Revision
http://www.normas9000.com/iso-9000-37.html

Parametrico
http://www.plataformaarquitectura.cl/cl/02-118243/%25c2%25bfque-es-el-diseno-parametrico


BDD
http://eamodeorubio.github.io/bdd-with-js/#
http://adrianmoya.com/2013/01/introduccion-a-desarrollo-guiado-por-comportamiento-bdd/
http://pioneerjs.com/
https://keyholesoftware.com/2015/08/31/atdd-cucumber-gherkin/
http://agilecoach.typepad.com/agile-coaching/2012/03/bdd-in-a-nutshell.html



HIPOTESIS y SUBs


Problema: El proceso de diseño no es transparente lo que producen impresiciones entre el cliente y el profesional de diseño.


Problema: 
El intercambio de directivas de diseño basado en reuniones cara a cara, mensajes de texto (e-mails, documentos, etc) 
e imágenes de referencia con el objetivo de diseñar un determinado producto causa demoras e impresiciones
en el desarrollo del diseño.
Hemos observado que un factor crítico es poder aplicar las directivas de diseño en un prototipo y mostrar el avance lo 
más pronto posible para que el cliente lo evalúe a fin de no continuar desarrollando un producto defectuoso. Además 
hemos observado que muchas veces las diferencias son ajustes menores ¿Cómo podríamos mejorar la exactitud en el intercambio 
de las directivas de diseño, proporcionar una visualización rápida y a la vez facilitar el ajuste menor con el
fin de disminuir los tiempos/costos de desarrollo del diseño?


Hipotesis: Creemos que desarrollando una aplicacion para que los clientes y diseñador puedan intercambiar los 
parámetros de diseño conseguiremos disminuir las impresiciones del desarrollo del diseño, facilitar su 
visualización y en caso de error facilitar la posibilidad de realizar un ajuste menor o bien deshacer el último cambio.
Sabremos que esto es cierto cuando disminuya los tiempos de diseño (y re-diseño), disminuya el estress del
diseñador al momento de enfrentarse a los cambios y se obtengan clientes satisfechos con el diseño del producto.

SubHipótesis:
- Invitar a los diferentes participantes
    - Enviar invitación por mail.
- Compartir link (par los anonimo)
    - Copiar el enlace
    - Enviar por mail
- historico general del proyecto (diseñador & personas sin conocimiento)
    - Ver cualquier instancia del proyecto.
    - Retomar una instancia antigua del proyecto.
- Ingreso / modificación del código para modelado booleano. (diseñador)
    - subir un archivos "código fuente"
    - editar el código fuente en línea.
    - descargar una copia del código (openjscad).
- Visualización del objeto (diseñador & personas sin conocimiento)
    - zoom
    - rotar
    - trasladar
- Ingreso  /  Modificación  de  parámetros   (diseñador & personas sin conocimiento)
    - text
    - slides
    - numeros
    - color
    - fecha
    - (ver si hay mas)
- Visor  de  comentarios  /  notas por modelo (diseñador & personas sin conocimiento)
    - Borrar el comentario
    - Editar el comentario
    - adjuntar archivos - Ingreso de archivos relacionados (diseñador & personas sin conocimiento)
    - Mensionar a uno o varios usuarios
    - hashtag 
- exportar archivos (Descargar Modelo)
    - a formato STL

http://revista.uxnights.com/lean-ux-parte-3/
 
Suposiciones de usuario:
1. ¿Quién es el usuario?
- Profesionales encargados de crear diseños de productos. Posee lascapacidades técnicas y la experiencia sobre procesos y metodolog ́ıas parallevar a cabo un proyecto de diseño en conjunto con el cliente/usuario.
- Personas sin conocimientos específicos. Interesados en fabricar productos o testear su funcionamiento, pueden ser emprendedores, artistas

2. ¿Cómo encaja nuestro producto en su vida o en su trabajo?
- Provee una plataforma de comunicación entre el profesional y el cliente. Además facilita el desarrollo del diseño
y el control de los cambios del diseño paso a paso.

3. ¿Que problemas soluciona nuestro producto?
- Facilita el intercambio de información relacionado al diseño del producto.
- Posibilita la visualización temprana del prototipo del producto.
- Posibilita la visualización paso a paso en el proceso de diseño.
- En caso de error, posibilita descartar un determinado avance en el prototipo.
- En caso de error, facilita el ajuste de ciertos parámetros del diseño.
- Posibilita ver el histórico de cambios y visualizar la diferencia entre ellas.

3. ¿Cuándo y cómo se utiliza nuestro producto?

4. ¿Cuáles son sus funciones más importantes? 

5. ¿Qué aspecto debe tener y cómo debe comportarse nuestro producto?



#\subsection{API RESTFUL}
#ver PDF: API REST y sistema de aprovisionamiento en containers para servIoTicy

#book: https://www.amazon.es/RESTful-API-Design-Practices-API-University-ebook/dp/B01L6STMVW

#Tesis Doctoral: #http://www.ics.uci.edu/~fielding/pubs/dissertation/rest_arch_style.htm
Artículo: https://www.infoq.com/articles/subbu-allamaraju-rest
Book: https://pages.apigee.com/rs/apigee/images/api-design-ebook-2012-03.pdf (ver fuentes)
Book: https://doc.lagout.org/programmation/Webservers/REST%20API%20Design%20Rulebook%20-%20Masse%20-%20O%27Reilly%20%282012%29/REST%20API%20Design%20Rulebook%20-%20Masse%20-%20O%27Reilly%20%282012%29.pdf

#Consideraciones de diseño 
  - no verbos
  - url en minuscula, no usar _ 
  - usar uri
  
#Metodos get/post/delete/etc


#\subsection{Programación SPA (Single Page App)}

#\begin{displayquote}
An SPA is an application delivered to the browser that doesn’t reload the page during use. Like all applications, it’s intended to help the user complete a task, such as “write a document” or “administer a web server.” We can think of an SPA as a fat client that’s loaded from a web server. \cite{Mikowski2015}
#\end{displayquote}

#\say{The single-page web interface is composed of individual components which can be updated/replaced independently, so that the entire page does not need to be reloaded on each user action.\cite{Mesbah2007}}



#\subsection{Lenguaje Ubicuo: De Historias a Escenarios y tests}

Scenario: Listar los Trabajos\vskip0mm
  Given: Estoy en URL\vskip0mm
  Then: Veo el título "Pieza de Pruebas"\vskip10mm
 
Scenario: Acceder al Escritorio de Trabajo\vskip0mm
  Given: Estoy en URL\vskip0mm
  When: Hago click en "Pieza de Pruebas"\vskip0mm
  Then: Estoy en URL\vskip0mm
  Then: Veo el título "Pieza de Pruebas"\vskip0mm
  And: Veo la etiqueta "Alto"\vskip0mm
  And: Veo la etiqueta "Ancho"\vskip0mm
  And: Veo la etiqueta "Profundidad"\vskip0mm
  And: Veo el Visor de la Pieza "stl-view"
  
  
  
  
  
  
  --------
  
  \subsection{Especificación BDD Escenarios, Cucumber, etc}

\subsection{Visualización y datos (WebGL, OpenJSCAD)}


\section{Métricas, ecuaciones, fórmulas, posibilidades}


\section{Construcción del sistema}

\subsection{Modelado de Classes}

\subsection{Diseño y Código en JS (FLUX?)}

\subsection{API: Loopback}

\section{Funcionalidades}

\subsection{Control de versiones}

\subsection{Integración con otras plataformas (embed)}

\subsection{Compartir proyectos (share)}

\subsection{Exportar modelos}

\subsection{Extras: Anotaciones, comentarios, adjuntar imagen, chat, etc}
