%-----------------------------------------------------
% Chapter 1: Introducción
%-----------------------------------------------------
\chapter{Introducción}
\label{chap:cap1}

%This is the introduction to the %thesis.\footnote{And this is a %footnote.}  The conclusion is in %Chapter \ref{chap:cap1} on page %\pageref{chap:cap1}.

\section{Conceptos Preliminares}
%Figure \ref{us_figure} shows the logo for the %University of Sussex.\footnote{This is a URL: %\url{http://www.sussex.ac.uk}} This is %consistent with Special Relativity %\citep{Einstein1905}. $E=mc^2$.

%\begin{figure}
%\centering
%\includegraphics[width=5cm]{uslogo}
%\caption[US Logo (optional short %caption)]{\label{us_figure} The logo for the %University of Sussex.}
%\end{figure}

La fabricación digital es una realidad gracias a la masificación de la tecnologías basadas en el \textbf{Control Numérico Computarizado} (CNC) \footnote{ https://es.wikipedia.org/wiki/Control\_numérico }, estas tecnologían en combinación con técnicas de \textbf{Diseño Asistido por Computadora} y  \textbf{Fabricación Asistida por Computadora} en inglés \textit{Computer-Aided Design/Computer-Aided Manufacturing} (CAD/CAM) se pueden aplicar a casi cualquier caso de manufactura  \citep{RojasLazo2006}. Sus aplicaciones son muy variadas: desde prototipos de productos en plástico hasta la fabricación robótica de prótesis médicas. %\footnote{ http://www.centroidcnc.com/digitizing.htm }

Conceptos como \textbf{Diseño Colaborativo de Productos} en inglés \textit{Collaborative Product Development} (CPD) \citep{Ruiz} hacen posible la participación de ingenieros, diseñadores industriales, arquitectos, programadores, artistas e incluso personas sin formación específica, generando nuevas técnicas de diseño y procedimientos de fabricación innovadoras.
Este escenario es conocido como \textbf{co-diseño} o \textbf{diseño colaborativo} el cuál se define como \textquote{\textit{un proceso de colaboración creativa entre diseñadores y personas, los cuáles no poseen una formación previa en diseño, con el propósito de resolver problemas significativos}} \citep{PerezGarcia2014}. \vskip
Las nuevas tendencias en la tecnología han ayudado a democratizar la creatividad y la participación en el diseño en muchos niveles. Los teléfonos inteligentes son un ejemplo de la tecnología al servicio de todas las personas, porque ofrecen oportunidades para la co-creación mediante sus aplicaciones \citep{Huerta2013}.
\vskip

En este contexto, la revisión del diseño es una etapa fundamental para el control de calidad de los productos porque permite evaluar la capacidad de los resultados obtenidos para cumplir los requisitos, identificar cualquier problema y proponer soluciones. 
Según la norma ISO 9001:2000 \textquote{\textit{revisar el diseño, o parte de él, son aquellas actividades que se realizan normalmente en una reunión de todos, o parte, de los responsables del diseño. Los resultados de las revisiones deben ser registradas. Lo más habitual es registrar estos resultados en actas de reunión o bien en algún formato que hayamos diseñado específicamente para controlar el diseño.}} \citep{Pereiro2005}.

El uso de enfoques o metodologías incentiva la participación en los procesos colaborativos, por ejemplo: \textquote{\textit{El enfoque \textbf{LEAN UX} propone principios básicos inspirados en las metodologías ágiles de desarrollo de software, uno de estos principios explica que la colaboración entre los participantes es más importante que la negociación de contratos. Si las decisiones se toman por consenso se obtiene como resultado iteraciones más rápidas y una verdadera implicación de todos los participantes en el proceso de diseño}} \citep{Gothelf2013}.

Los diseñadores industriales Tek-Jin Nam y Kyung Sakong explican que los ambientes para el diseño 3D colaborativo son diferentes a otros ambientes colaborativos en términos de comunicación.  \textquote{\textit{El contenido compartido en los ambientes de diseño 3D colaborativos por lo general implica el uso de modelos 3D como medios de comunicación entre los participantes con el fin de visualizar ideas abstractas y se usan iterativamente durante todo el proceso de diseño.}} \citep{Tek-JinNam2009}. %Además, para permitir la participación es necesario contar un repositorio común de datos además de la geometría, incluyendo especificaciones y documentos en diferentes formatos según las particularidades de cada proyecto.  
\vskip
Los trabajos relacionados con la colaboración de diseños 3D se consideran sistemas de diseño \textit{basados en la visualización}, donde la atención se centra en revisar y manipular un modelo 3D virtual.
Sin embargo, en la tarea de construcción de estos sistemas se presentan diversos desafíos técnicos, relacionados en gran medida a los requerimientos de crear, editar, expresar y revisar los modelos 3D. 
\vskip
Considerando los usos y hábitos actuales de los usuarios de internet, en este trabajo de investigación se describe el proceso de creación de una herramienta tecnológica para dar soporte al co-diseño utilizando metodologías acordes a los principios colaborativos planteados. \vskip

\clearpage
\section{Objetivos}
\subsubsection {Objetivo General}
Diseñar e implementar un prototipo de software colaborativo multiplataforma para revisiones de modelos 3D paramétricos.

\subsubsection {Objetivos Específicos}
\begin{itemize}
  \item Describir el estado actual de la tecnologías colaborativas para el diseño iterativo de modelos 3D.
  \item Analizar y describir las tecnologías web disponibles para la visualización y manipulación de modelos 3D.
  \item Diseñar una Interfaz Gráfica de Usuario (GUI) de un entorno colaborativo basándose en el diseño de experiencia de usuario UX con el enfoque LEAN UX.
  \item Determinar los requerimientos de software necesarios para diseñar el entorno colaborativo que permita la revisión de modelos 3D y genere archivos preparados para la fabricación digital.
  \item Diseñar un modelo de dominio de una capa de servicios utilizando metodologías ágiles.
  \item Desarrollar un prototipo de software con herramientas FLOSS según los requerimientos de la GUI y el Modelo de dominio.
  \item Probar el prototipo con usuarios finales en dos escenarios diferentes y evaluar los resultados.
\end{itemize}





