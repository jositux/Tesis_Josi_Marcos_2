%-----------------------------------------------------
% Chapter 1: Introducción
%-----------------------------------------------------
\chapter{Introducción}
\label{chap:cap1}

%This is the introduction to the %thesis.\footnote{And this is a %footnote.}  The conclusion is in %Chapter \ref{chap:cap1} on page %\pageref{chap:cap1}.

\section{Conceptos Preliminares}
%Figure \ref{us_figure} shows the logo for the %University of Sussex.\footnote{This is a URL: %\url{http://www.sussex.ac.uk}} This is %consistent with Special Relativity %\citep{Einstein1905}. $E=mc^2$.

%\begin{figure}
%\centering
%\includegraphics[width=5cm]{uslogo}
%\caption[US Logo (optional short %caption)]{\label{us_figure} The logo for the %University of Sussex.}
%\end{figure}

La fabricación digital es una realidad gracias a la masificación de la tecnologías basadas en Control Numérico Computarizado (CNC) \footnote{ https://es.wikipedia.org/wiki/Control\_numérico } 
en combinación con técnicas de Diseño Asistido por Computadora y  Fabricación Asistida por Computadora en inglés Computer-Aided Design/Computer-Aided Manufacturing (CAD/CAM) \citep{RojasLazo2006}, se pueden aplicar a casi cualquier caso de manufactura. Sus aplicaciones son muy variadas: desde prototipos de productos en plástico a la fabricación robótica de prótesis dentales. \footnote{ http://www.centroidcnc.com/digitizing.htm }

Conceptos como Diseño Colaborativo de Productos del inglés \textit{Collaborative Product Development} (CPD) (\citeauthor{Ruiz}, \citeyear{Ruiz})  hacen posible la participación de ingenieros, diseñadores industriales, arquitectos, programadores, artistas e incluso personas sin formación específica, generando nuevas técnicas de diseño y procedimientos de fabricación innovadoras.
Este escenario es conocido como co-diseño o diseño colaborativo, el cuál puede definirse como \textquote{un proceso de colaboración creativa entre diseñadores y personas, los cuáles no poseen una formación previa en diseño, con el propósito de resolver problemas significativos.} (\citeauthor{PerezGarcia2014}, \citeyear{PerezGarcia2014}) \vskip

\textquote{Las nuevas tendencias en la tecnología han ayudado a democratizar la creatividad y la participación en
el diseño, en muchos niveles. Los teléfonos inteligentes son un ejemplo de la tecnología al servicio de todas las personas, porque ofrecen oportunidades para la co-creación mediante sus aplicaciones.}\citep{Huerta2013}
\vskip
\textquote{El contenido compartido en los entornos de diseño colaborativos por lo general implica el uso de modelos 3D como medios de comunicación entre los participantes con el fin de visualizar ideas abstractas y se usan iterativamente durante todo el proceso de diseño. Además, para permitir la participación es necesario contar un repositorio común de datos además de la geometría, incluyendo especificaciones y documentos en diferentes formatos según las particularidades de cada proyecto.  (\citeauthor{Tek-JinNam2009}, \citeyear{Tek-JinNam2009})}
\vskip
La revisión del diseño es una etapa fundamental para el control de calidad porque permite evaluar la capacidad de los resultados obtenidos para cumplir los requisitos, identificar cualquier problema y proponer soluciones. Los resultados de las revisiones deben ser registrados, lo más habitual es registrar estos resultados en actas de reunión o bien en algún formato de cláusula o contrato que se haya diseñado específicamente para controlar el diseño. 
\footnote{ http://www.portalcalidad.com/articulos/52-diseno\_productos\_iso\_9001 } 

Para satisfacer a los participantes en el proceso de revisión, el enfoque LEAN UX (\citeauthor{Gothelf2013}, \citeyear{Gothelf2013}) propone principios básicos inspirados en las metodologías ágiles de desarrollo de software, uno de ellos explica que la colaboración entre los participantes es más importante que la negociación de contratos. Si las decisiones se toman por consenso se obtiene como resultado iteraciones más rápidas y una verdadera implicación de todos en el diseño.

En los siguientes capítulos, se define el proceso de creación de una herramienta tecnológica que soporta el co-diseño, utilizando metodologías de diseño y desarrollo de software con un enfoque acorde a los principios colaborativos planteados. \vskip

\clearpage
\section{Objetivos}
\subsection {Objetivo General}
Diseñar e implementar un prototipo de software colaborativo multiplataforma para revisiones de modelos 3D paramétricos.

\subsection {Objetivos Específicos}
\begin{itemize}
  \item Describir el estado actual de la tecnologías colaborativas para el diseño iterativo de modelos 3D.
  \item Analizar y describir las tecnologías web disponibles para la visualización y manipulación directa de modelos 3D.
  \item Diseñar una Interfaz Gráfica de Usuario (GUI) de un entorno colaborativo basándose en el diseño de experiencia de usuario UX con el enfoque LEAN UX.
  \item Diseñar un modelo de dominio de una capa de servicios utilizando BDD (Desarrollo guiado por comportamiento) como metodología ágil.
  \item Determinar los requerimientos de software necesarios para diseñar el entorno colaborativo que permita la revisión de modelos 3D y genere archivos preparados para la fabricación digital.
  \item Desarrollar un prototipo de software con herramientas FLOSS según los requerimientos de la GUI y el Modelo de dominio.
  \item Probar el prototipo con usuarios finales en dos escenarios diferentes y evaluar los resultados.
\end{itemize}





